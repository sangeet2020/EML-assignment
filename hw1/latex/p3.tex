\documentclass{article}[a4paper]
\usepackage[a4paper, left=2.5cm,right=2cm,top=2.5cm,bottom=2.5cm]{geometry}
\usepackage[utf8]{inputenc}
\usepackage{csquotes}
\usepackage{booktabs}
\usepackage{amsmath}
\usepackage{mathtools}% http://ctan.org/pkg/mathtools
\usepackage{caption}
\captionsetup{width=.75\textwidth}
\usepackage[usestackEOL]{stackengine}
\usepackage{float}
\usepackage{subcaption}
\usepackage{tikz}
\usetikzlibrary{shapes.geometric,arrows,positioning,fit}
\usetikzlibrary{shapes,calc,arrows}
\usepackage{natbib}
\usepackage{graphicx}
\usepackage{subfiles}
\usepackage{blindtext}
\usepackage{hyperref}
\usepackage{float}
\usepackage{physics} % for 'pdv' macro
\usepackage{qtree}
\usepackage{stmaryrd}
\usepackage{multicol}
\usepackage{xparse}
\usepackage{amssymb}
\usepackage{xcolor}
\usepackage{neuralnetwork}
\usepackage{pgfplots}
\usepackage{sidecap}
 \usepackage{xcolor}
\usepackage{tikz}
\usetikzlibrary{decorations.pathreplacing}

\pgfplotsset{compat=1.16}

% Syntax: \colorboxed[<color model>]{<color specification>}{<math formula>}
\newcommand*{\colorboxed}{}
\def\colorboxed#1#{%
  \colorboxedAux{#1}%
}
\newcommand*{\colorboxedAux}[3]{%
  % #1: optional argument for color model
  % #2: color specification
  % #3: formula
  \begingroup
    \colorlet{cb@saved}{.}%
    \color#1{#2}%
    \boxed{%
      \color{cb@saved}%
      #3%
    }%
  \endgroup
}

\def\XXX#1{\textcolor{red}{XXX #1}}

\NewDocumentCommand{\codeword}{v}{%
\texttt{\textcolor{blue}{#1}}%
}
\newcommand\Mycomb[2][^n]{\prescript{#1\mkern-0.5mu}{}C_{#2}}
\def\XXX#1{\textcolor{red}{XXX #1}}
\newcommand{\vect}[1]{\boldsymbol{\textcolor{blue}{#1}}}

\title{\textbf{Elements of Machine Learning}\\
Assigment 1 - Problem 3
}
\author{ Sangeet Sagar(7009050), Philipp Schuhmacher(7010127)\\
        \texttt{\{sasa00001,phsc00003\}@stud.uni-saarland.de}
}
% \pgfplotsset{compat=1.17}
% \parindent 0in
% \parskip 0em
\begin{document}
\maketitle
\section*{Gauss-Markov theorem}
\subsection*{Part 1}
\textcolor{red}{Explain in your own words what the Gauss-Markov theorem is and why it is important.}\\
Gauss-Markov theorem states that the least squares estimator is the best unbiased estimate in a linear regression model. There is practically no other linear unbiased estimator with smaller variance. In this the error expectation is null and are uncorrelated with zero variance.\\
The Gauss-Markov model is important because when applied in a linear regression model, we can find the best unbiased estimator if Markov assumptions apply.

\subsection*{Part 2}
\textcolor{red}{Explain in your own words and write down the formal math formulation of the three \textit{Gauss-Markov error term assumptions}.}\\
\begin{enumerate}
    \item All error terms have a mean of exactly 0. 
    \begin{align*}
        E[\epsilon_i] = 0; i=0,1,\hdots,n
    \end{align*}
    \item Errors are not dependent on one another or are not correlated.
    \begin{align*}
        Cov(\epsilon_i, \epsilon_j) = 0; i,j=0,1,\hdots,n \text{ and } i \neq j
    \end{align*}
    \item The variance of the error terms is constant and finite, regardless of how the variables look. So to speak ``homoscedastic''
    \begin{align*}
        Var(\epsilon_i) = \sigma^2; i=0,1,\hdots,n
    \end{align*}
\end{enumerate}

\subsection*{Part 3}
\textcolor{red}{Is it also the \textit{best} (in terms of test error) linear unbiased estimate (argue with the bias-variance trade-off)?}\\
Yes, Gauss-Markov model provides the best linear unbiased estimate for least-square fit. That is to say the estimates have the smallest variance among all unbiased estimates. For a least square model, it suffers from high variance and given that estimates are unbiased, they are mostly non-zero, making the model difficult to interpret. With Gauss Markov theorem, we get unbiased estimates, but with least variance.
% If we want to find the best estimator for the test error, then maybe there is a better one. You accept to be more biased, but to have less variance. So-called bias-variance tradeoff. That depends entirely on $f$

\bibliographystyle{plainnat}
\bibliography{references}
\end{document}